\documentclass[a4paper,12pt,oneside]{book}
\usepackage[UTF8]{ctex}
\usepackage[english]{babel}
\usepackage[T1]{fontenc}
\usepackage{graphicx}
\usepackage{amsmath}
\usepackage{listings}
\usepackage{color}
\usepackage{parskip}
\usepackage{soul}

\setlength{\parindent}{0pt}
\definecolor{dkgreen}{rgb}{0,0.6,0}
\definecolor{gray}{rgb}{0.5,0.5,0.5}
\definecolor{mauve}{rgb}{0.58,0,0.82}
\lstset{language=Python,
frame=tb,
aboveskip=3mm,
  belowskip=3mm,
  showstringspaces=false,
  columns=flexible,
  basicstyle={\small\ttfamily},
  numbers=none,
  numberstyle=\tiny\color{gray},
  keywordstyle=\color{blue},
  commentstyle=\color{dkgreen},
  stringstyle=\color{mauve},
  breaklines=true,
  breakatwhitespace=true,
  tabsize=3}
\pagestyle{headings}
\begin{document}
\chapter{Sample Chapter}

\section{Neural Network}
\paragraph{变量表示:}
在看本文之前你可能已经对Neural Network有了一定的了解,或者已经听说过。Neural Network可以抽象的表示为下图:\\
\begin{center}
    \includegraphics{1NeuralNetwork.jpg}\\
\end{center}
\begin{center}
    【图1】\\
\end{center} 
Neural Network的最基本组成单位是Neuron,在图中用一个圆圈表示。
该图所表示的是一个有两个Hidden Layer的Neural Network。该Neural Network的层数为5层。
接下来我们定义一下变量:\\
相邻两个Layer的节点之间通过Synapse相连接,每个Synapse上有一个权重Weight。
$w_{kj}^l$表示第$(l-1)$层的第个$k$个neuron连接到第$l$层的第$j$个neuron的权重;\\
$b_j^l$ 表示第$l$层的第$j$个neuron的bias;\\
$z_j^l$ 表示第$l$层的第$j$个neuron的输入,即:$z_l^j=\sum_k {w_{kj}^l a^{l-1}_k+b^l_j}$ \\
$a_j^l$ 表示第$l$层的第$j$个neuron的输出,即:$a^j_l=\sigma(\sum_k {w_{kj}^l a^{l-1}_k+b_j^l)}$ \\
其中,$\sigma$ 表示Activation Function\\
有了变量的定义之后,我们还需要定义Loss Function。比较常见的Loss Function是Quadratic Loss Function:\\
\begin{center}
    $C = \frac{1}{2n}\sum_x {||y(x) - a^L(x)||^2}$\\
\end{center}
其中,$x$表示输入的样本,$y$表示实际的分类情况,$a^L$表示预测的分类情况,
$L$表示神经网络的最大层数。\\
\paragraph{Neural Network计算案例:}
假设我们有这样一个数据:
\begin{equation*}
    X = 
    \begin{bmatrix}
        0 & 0 & 1\\
        1 & 1 & 1\\
        1 & 0 & 1\\
        0 & 1 & 1\\
    \end{bmatrix}
    , Y = \begin{bmatrix}
        0 & 1 & 1 & 0 
    \end{bmatrix}^\top
\end{equation*}
\begin{lstlisting}
    # Python Code:
    # Generate Input and Output data:
    import numpy as np
    X = np.array([[0, 0, 1],[1, 1, 1],[1, 0, 1],[0, 1, 1]])  # 4 by 3 matrix
    Y = np.array([[0, 1, 1, 0]]).T  # 4 by 1 matrix
\end{lstlisting}

如何将上面的Neural Network应用到这个数据上呢:\\
1. 首先我们先选择Activation Function为Sigmoid Function。即,$\sigma{(x)}=\frac{1}{1+e^{-x}}$。\\
该激活函数对$x$的导函数是:\\
$\sigma^{'}(x)=\frac{e^{-x}}{(1+e^{-x})^2}\\
=\frac{1+e^{-x}-1}{(1+e^{-x})^2}\\
=\frac{1}{1+e^{-x}}-\frac{1}{(1+e^{-x})^2}\\
=\sigma(x)\dot{(1-\sigma{(x)})}$\\
\noindent
\begin{lstlisting} 
    # Python Code:
    def sigmoid(x, derive=False):
        if derive == False:
            return 1 / (1 + np.exp(-x))
        else:
            return x * (1 - x)
\end{lstlisting}
\noindent
2. 接着,我们定义神经网络的结构(见【图2】),\textbf{自变量\textit{X}的每一行就是一个training example。每一列就是一个Input Neuron。所以这里我们有4个training examples和3个Input Neuron。
这里,由于我们的Neural Network一共有3层,所以需要两个随机权重矩阵,分别用\textit{$W^1$}和\textit{$W^2$}表示。\textit{$W^1$}是一个$3\times4$的矩阵。由于Y只有一列,所以\textit{$W^2$}是一个$4\times1$的矩阵}
第$(l-1)$层的第$k$个neuron与第$l$层的第$j$个neuron的权重用$w^l_{kj}$表示。\\
\begin{equation*}
    W^1 = \begin{bmatrix}
        -0.01908238 & -0.54517074 & -0.49128704 & -0.88394168\\
        -0.13116675 & -0.37640824 & 0.39268698 & -0.24449632\\
        -0.64079264 & -0.95064254 & -0.86550074 & 0.35878555\\
    \end{bmatrix}
    , W^2 = \begin{bmatrix}
        -0.09260631\\
        0.07315842\\
        0.79334259\\
        0.98067789\\
    \end{bmatrix}
\end{equation*}
\begin{lstlisting}
    # Python Code:
    random.seed(0)
    # W1 is Weight Matrix between Input Layer and Hidden Layer:
    W1 = 2 * np.random.random((3,4)) - 1  # 3 by 4 matrix
    # W2 is Weight Matrix between Hidden Layer and Output Layer:
    W2 = 2 * np.random.random((4,1)) - 1  # 4 by 1 matrix
\end{lstlisting}
我们看上面的Code Block,W1是
(权重矩阵一般随机初始化,比如从0均值的均匀分布或高斯分布中采样得到。如果我们将权重矩阵初始化成$[100.41, 100.1, 100.0]$,你会发现训练后的Neural Network的权重没有变化,说明这种权重矩阵初始化的方法不适合。)\\
3. 有了这些后,我们就可以建立一个for loop来训练Neural Network了。

\begin{lstlisting}
    # Python Code:
    for i in range(10000): 
        a1 = X  
        # The 1st Layer is the Input Layer, X.
        # a1 --> 4 by 3 matrix

        z2 = np.dot(a1, W2)  
        # z2 --> 3 by 3 matrix.
        # W2 --> 3 by 4 matrix

        a2 = sigmoid(z2, derive=False)
        # a2 --> 4 by 4 matrix
        # Here, we finished calculation before Hidden Layer
        
        z3 = np.dot(a2, W3)
        # W3 --> 4 by 1 matrix
        # z3 --> 4 by 1 matrix

        a3 = sigmoid(z3, derive=False)
        # a3 --> 4 by 1 matrix

        # Here, we finished calculation of forward feed.


        # Then, we'll calculate Back Propogation:
        e3 = Y - a3  # The Error from Output Layer.
        # e3 --> 4 by 1 matrix
        
        delta3 = e3 * sigmoid(a3, derive=True)
        # delta3 --> 4 by 1 matrix

        e2 = delta3.dot(W2.T)  # The Error from Hidden Layer.
        # e2 --> 4 by 1 matrix

        delta2 = e2 * sigmoid(a2, derive=True)
        #

        # Gradient Descent (Update Weight Matrices):
        W3 = W3 + a2.T.dot(delta3)
        W2 = W2 + a1.T.dot(delta2)
    print('Output after training: ', a3)

>>> Output after training:  
[[0.00517135]
[0.99506876]
[0.99585861]
[0.00392782]]
\end{lstlisting}

4. 如果你已经读懂了上面的部分,那么可以跳过这一段。如果没懂的话,我们还是用上面的案例,用数学的方式计算一下这个Neural Network:
在计算之前,如果你已经运行了上面的代码,能发现第一次for循环的结果为(你的结果可能与我的不同,但不要紧。我们只关心最后的结果):
\begin{lstlisting}
>>>Output after training:  
[[0.43888284]
 [0.47587257]
 [0.46470967]
 [0.45109288]]
\end{lstlisting}
首先我们有自变量\textit{X}和因变量\textit{Y}:
\begin{equation*}
    X = \begin{bmatrix}
        x_{11} & x_{12} & x_{13} \\
        x_{21} & x_{22} & x_{23} \\
        x_{31} & x_{32} & x_{33} \\
        x_{41} & x_{42} & x_{43} \\
    \end{bmatrix}
    = \begin{bmatrix}
        0 & 0 & 1\\
        1 & 1 & 1\\
        1 & 0 & 1\\
        0 & 1 & 1\\
    \end{bmatrix}
    , Y = \begin{bmatrix}
        y_{11}\\
        y_{12}\\
        y_{13}\\
        y_{14}\\
    \end{bmatrix}
    = \begin{bmatrix}
        0 \\
        1 \\
        1 \\
        0 \\
    \end{bmatrix}
\end{equation*}
以及最初随机生成的两个权重矩阵\textit{$W^1$}和\textit{$W^2$}:
\begin{equation*}
    W^1 = \begin{bmatrix}
        w^1_{11} & w^1_{12} & w^1_{13} & w^1_{14} \\
        w^1_{21} & w^1_{22} & w^1_{23} & w^1_{24} \\
        w^1_{31} & w^1_{32} & w^1_{33} & w^1_{34} \\
    \end{bmatrix}
    = \begin{bmatrix}
        -0.01908238 & -0.54517074 & -0.49128704 & -0.88394168\\
        -0.13116675 & -0.37640824 & 0.39268698 & -0.24449632\\
        -0.64079264 & -0.95064254 & -0.86550074 & 0.35878555\\
    \end{bmatrix}
\end{equation*}
\begin{equation*}
    W^2 = \begin{bmatrix}
        w^2_{11}\\
        w^2_{21}\\
        w^2_{31}\\
        w^2_{41}\\
    \end{bmatrix}
    = \begin{bmatrix}
        -0.09260631\\
        0.07315842\\
        0.79334259\\
        0.98067789\\
    \end{bmatrix}
\end{equation*}
另外要注意的一点是,我们通常将误差矩阵\textit{B}初始化为全0矩阵,即:
\begin{equation*}
    B^1 = \begin{bmatrix}
        b^1_{11} & b^1_{12} & b^1_{13} & b^1_{14} \\
        b^1_{21} & b^1_{22} & b^1_{23} & b^1_{24} \\
        b^1_{31} & b^1_{32} & b^1_{33} & b^1_{34} \\
        b^1_{41} & b^1_{42} & b^1_{43} & b^1_{44} \\
    \end{bmatrix}
    = \begin{bmatrix}
        0 & 0 & 0 & 0 \\
        0 & 0 & 0 & 0 \\
        0 & 0 & 0 & 0 \\
        0 & 0 & 0 & 0 \\
    \end{bmatrix}
\end{equation*}
那么对于神经网络来说,第一层(i.e. Input Layer)就是\textit{X};最后一层(i.e. Output Layer)就是\textit{Y};Hidden Layer只有一\\
4.1. 接下来计算从Input Layer到Hidden Layer的传播过程:
我们用$z^l_{kj}$表示从第$(l-1)$层第$k$个neuron到第$l$层第$j$个neuron的输入值。特别地,\textit{X}=$z^1$。\\
另外,我们用$h^l_{kj}$表示从第$(l-1)$层第$k$个neuron到第$l$层第$j$个neuron的输出值。\\
第$(l-1)$层的输出值$h^{l-1}_{kj}$经过Activation Function之后,就得到了第$l$层的输入值$z^l{kj}$
对于从Input Layer上的第$k$个neuron传向Hidden Layer上的第$j$个neuron的输出值$h^2_{kj}$有:\\
\begin{equation*}
    h^2_{kj}=h^1\cdot W^1 + B^1 = \textit{X} \cdot W^1 + B^1\\
    = \begin{bmatrix}
        x^1_{11} & x^1_{12} & x^1_{13} \\
        x^1_{21} & x^1_{22} & x^1_{23} \\
        x^1_{31} & x^1_{32} & x^1_{33} \\
        x^1_{41} & x^1_{42} & x^1_{43} \\
    \end{bmatrix} \cdot
    \begin{bmatrix}
        w^1_{11} & w^1_{12} & w^1_{13} & w^1_{14} \\
        w^1_{21} & w^1_{22} & w^1_{23} & w^1_{24} \\
        w^1_{31} & w^1_{32} & w^1_{33} & w^1_{34} \\
    \end{bmatrix} 
\end{equation*}
\begin{equation*}
    = \begin{bmatrix}
        0 & 0 & 1\\
        1 & 1 & 1\\
        1 & 0 & 1\\
        0 & 1 & 1\\
    \end{bmatrix} \cdot
    \begin{bmatrix}
        -0.01908238 & -0.54517074 & -0.49128704 & -0.88394168\\
        -0.13116675 & -0.37640824 & 0.39268698 & -0.24449632\\
        -0.64079264 & -0.95064254 & -0.86550074 & 0.35878555\\
    \end{bmatrix}    
\end{equation*}
\begin{equation*}
    = \begin{bmatrix}
        -0.64079264 & -0.95064254 & -0.86550074 & 0.35878555\\
        -0.79104177 & -1.87222152 & -0.9641008 & -0.76965245\\
        -0.65987502 & -1.49581328 & -1.35678778 & -0.52515613\\
        -0.77195939 & -1.32705078 & -0.47281376 & 0.11428923\\
    \end{bmatrix}    
\end{equation*}
4.2. 接着,我们用Sigmoid Function将$h^2_{kj}$激活后得到$z^2_{kj}$:
\begin{equation*}
    z^2_{kj} = \sigma(h^2_{kj})
    = \begin{bmatrix}
        \sigma(-0.64079264) & \sigma(-0.95064254) & \sigma(-0.86550074) & \sigma(0.35878555)\\
        \sigma(-0.79104177) & \sigma(-1.87222152) & \sigma(-0.9641008) & \sigma(-0.76965245)\\
        \sigma(-0.65987502) & \sigma(-1.49581328) & \sigma(-1.35678778) & \sigma(-0.52515613)\\
        \sigma(-0.77195939) & \sigma(-1.32705078) & \sigma(-0.47281376) & \sigma(0.11428923)\\
    \end{bmatrix}
\end{equation*}
\begin{equation*}
    = \begin{bmatrix}
        z^2_{11} & z^2_{12} & z^2_{13} & z^2_{14} \\
        z^2_{21} & z^2_{22} & z^2_{23} & z^2_{24} \\
        z^2_{31} & z^2_{32} & z^2_{33} & z^2_{34} \\
        z^2_{41} & z^2_{42} & z^2_{43} & z^2_{44} \\
    \end{bmatrix}
    = \begin{bmatrix}
        0.34506738 & 0.27875562 & 0.29619137 & 0.58874642\\
        0.31194502 & 0.13328488 & 0.2760579 & 0.31655429\\
        0.34076769 & 0.18305079 & 0.20476287 & 0.37164735\\
        0.3160554  & 0.20964762 & 0.38395048 & 0.52854125\\
     \end{bmatrix}
\end{equation*}
4.3. 得到$z^2_{kj}$后,我么将$z^2_{kj}$作为Output Layer的输入$h^2_{kj}$,来计算Output Layer处的输出$h^3_{kj}$。同样地,先计算:\\
\begin{equation*}
    h^3_{kj} = z^2_{kj} \cdot W^2 = \sigma(h^2_{kj}) \cdot W^2
    = \begin{bmatrix}
        z^2_{11} & z^2_{12} & z^2_{13} & z^2_{14} \\
        z^2_{21} & z^2_{22} & z^2_{23} & z^2_{24} \\
        z^2_{31} & z^2_{32} & z^2_{33} & z^2_{34} \\
        z^2_{41} & z^2_{42} & z^2_{43} & z^2_{44} \\
    \end{bmatrix} \cdot
    \begin{bmatrix}
        w^2_{11}\\
        w^2_{21}\\
        w^2_{31}\\
        w^2_{41}\\
    \end{bmatrix}
\end{equation*}
\begin{equation*}
    = \begin{bmatrix}
        0.34506738 & 0.27875562 & 0.29619137 & 0.58874642\\
        0.31194502 & 0.13328488 & 0.2760579 & 0.31655429\\
        0.34076769 & 0.18305079 & 0.20476287 & 0.37164735\\
        0.3160554  & 0.20964762 & 0.38395048 & 0.52854125\\
    \end{bmatrix} \cdot
    \begin{bmatrix}
        -0.09260631\\
        0.07315842\\
        0.79334259\\
        0.98067789\\
    \end{bmatrix}
\end{equation*}
\begin{equation*}
    = \begin{bmatrix}
       0.80078972\\
       0.51030912\\
       0.50874791\\
       0.80900175\\
    \end{bmatrix}
\end{equation*}
4.4. Forward Feed过程的最后一步就是将Activation Function放到最后一层的输出$h^3_{kj}$上:
\begin{equation*}
    z^3_{kj} = \sigma(h^3_{kj})
    = \begin{bmatrix}
        z^3_{11} \\
        z^3_{21} \\
        z^3_{31} \\
        z^3_{41} \\
    \end{bmatrix}
    = \begin{bmatrix}
        \sigma(h^3_{11}) \\
        \sigma(h^3_{21}) \\
        \sigma(h^3_{31}) \\
        \sigma(h^3_{41}) \\
    \end{bmatrix}
\end{equation*}
\begin{equation*}
    = \begin{bmatrix}
        \sigma(0.80078972) \\
        \sigma(0.51030912) \\
        \sigma(0.50874791) \\
        \sigma(0.80900175) \\
    \end{bmatrix}
    = \begin{bmatrix}
       0.69014339\\
       0.62487894\\
       0.62451291\\
       0.69189674\\
    \end{bmatrix}
\end{equation*}
到此,我们发现我们计算的输出和Python代码的第一次输出相同。因此完成了第一次的迭代过程。

4.5. 接下来进行Back Propagation:\\
Back Propogation过程中最重要的就是更新权重矩阵$W^l(old)$得到$W^l(new)$。而$W^l(new) = W^l(old) + a^{(l-1)_{kj}\space T} \cdot \delta^l$。\\
其中,$\delta^l$是第$l$层到第$(l-1)$层产生的错误。用$\delta^l = E^l \circ \sigma^{'}(a^{l}_{kj})$求得。其中$E^l$是Back Propogation过程中,第$l$层产生的Error。\\
\underline{\textbf{本文中,点积运算的两个矩阵之间用$\cdot$连接;哈达玛积运算的两个矩阵用$\circ$表示。}}\\
哈达玛积是两个矩阵的对应位置的元素之间进行乘法运算,可以用\lstinline{np.multiply(array1, array2)}来在Python中实现。\\
Error通常是用由Loss Function(或Cost Function)求导得来,记作$C$。
本例是分类问题,所以用$\frac{1}{2} \sum_{i=1}^{n}(Y_i-a^L_i)^2$计算作为Loss Function,其中$n$是样本个数。
\begin{equation}
    \begin{split}
        a^L &= \sigma(W^L\cdot a^{L-1}+b^L)\\
        a^{L-1} & = \sigma(W^{L-1} \cdot a^{L-2} + b^{L-1}) \\
        & = \sigma(W^L\cdot \sigma(W^{L-1} \cdot a^{L-2} + b^{L-1}) + b^L)\\
        \vdots \\
        a^1 &= \sigma(W^1 \cdot a^0 + b^1)\\
        a^0 &= X
    \end{split}
\end{equation}
在Back Propagation的过程中,如果是从最后一层$L$到倒数第二层$(L-1)$,那么$E^L=\frac{\partial{C}}{\partial{Y}}=Y-a^L$。同理,如果是计算从第$L$层到第$l$层的$Error^l$,则要用到Chain Rule,有$E^l=\frac{\partial{C}}{\partial{W^l}}=\frac{\partial{C}}{\partial{Y}}\cdot \frac{\partial{Y}}{\partial{W^l}}$。我们总结出如下规律:
\begin{equation}
    \begin{split}
        E^L &= \frac{\partial{C}}{\partial{a^L}}=a^L-Y \\
        \vdots\\
        E^l &= \frac{\partial{C}}{\partial{W^l}} = \frac{\partial{C}}{\partial{a^l}} \cdot \frac{\partial{a^l}}{\partial{W^l}} = (a^L-Y) \cdot \frac{\partial{a^l}}{\partial{W^l}}\\
        &= (a^L-Y) \cdot \sigma^{'}(W^l \cdot a^{l-1}+b^l)\cdot (a^{l-1})^T\\
        E^{l-1} &= \frac{\partial{C}}{\partial{W^{l-1}}} = \frac{\partial{C}}{\partial{a^{l-1}}} \cdot \frac{\partial{a^{l-1}}}{\partial{W^{l-1}}} =\frac{\partial{C}}{\partial{a^l}}\cdot\frac{\partial{a^l}}{\partial{a^{l-1}}}\cdot\frac{\partial{a^{l-1}}}{\partial{W^{l-1}}} \\
        &= \frac{\partial{C}}{\partial{a^l}}\cdot\frac{\partial{a^l}}{\partial{a^{l-1}}}\cdot \sigma^{'}(W^{l-1}a^{l-2}+b^{l-1})(a^{l-2})^T\\
        \vdots \\
   \end{split}
\end{equation}
由于我们的神经网络有三层(Input Layer,Hidden Layer,和Output Layer)所以在Back Propagation的过程中,从Output Layer到Hidden Layer产生的Error,$E^3=a^3-Y$(见公式1.2)。从Hidden Layer到Input Layer的Error:
\begin{equation}
    E^2=\frac{\partial{C}}{\partial{a^3}}\cdot\frac{\partial{a^3}}{\partial{a^2}}\cdot\frac{\partial{a^2}}{W^2}=(Y-a^3) \circ \sigma{'}(a^3) \cdot (W^3)^T
\end{equation}
\begin{enumerate}
    \item 从后往前计算每层的Error。首先计算Neural Network最后一层的Error,也就是代码中的 \lstinline{e3 = Y - a3},记作$E^{3}$。所以:
    \begin{equation*}
        E^3 = Y - z^{3}_{kj}
        = \begin{bmatrix}
            0 \\
            1 \\
            1 \\
            0 \\
        \end{bmatrix}
        - \begin{bmatrix}
           0.69014339\\
           0.62487894\\
           0.62451291\\
           0.69189674\\
        \end{bmatrix}
        = \begin{bmatrix}
            -0.69014339\\
            0.37512106\\
            0.37548709\\
            -0.69189674\\
        \end{bmatrix}
    \end{equation*}
    \item 与此同时从后往前计算每层的错误$\delta^{l}$等于当前层的Error $E^{l}$与激活函数的导数$\sigma^{'}_{z^l_{kj}}$的哈达玛积(Hadamard Product)。哈达玛积是两个矩阵的对应位置的元素之间进行乘法运算,可以用\lstinline{np.multiply(array1, array2)}来在Python中实现。\\
    \underline{\textbf{本文中,点积运算的两个矩阵之间用$\cdot$连接;哈达玛积运算的两个矩阵用$\circ$表示}}。那么最后一层的错误$\delta^3$:
    \begin{equation*}
        \delta^3 = E^{3} \circ \sigma^{'}(z^3_{kj})
        = \begin{bmatrix}
            -0.69014339\\
            0.37512106\\
            0.37548709\\
            -0.69189674\\
        \end{bmatrix} \circ
        \begin{bmatrix}
            \sigma^{'}(0.69014339)\\
            \sigma^{'}(0.62487894)\\
            \sigma^{'}(0.62451291)\\
            \sigma^{'}(0.69189674)\\
        \end{bmatrix}
        = \begin{bmatrix}
            -0.69014339\\
            0.37512106\\
            0.37548709\\
            -0.69189674\\
        \end{bmatrix} \circ
        \begin{bmatrix}
            0.21384549\\
            0.23440525\\
            0.23449654\\
            0.21317564\\
        \end{bmatrix}
    \end{equation*}
    \begin{equation*}
        = \begin{bmatrix}
            -0.69014339 \times 0.21384549\\
            0.37512106 \times 0.23440525\\
            0.37548709 \times 0.23449654\\
            -0.69189674 \times 0.21317564\\
        \end{bmatrix}
        = \begin{bmatrix}
            -0.14758405\\
            0.08793035\\
            0.08805042\\
            -0.14749553\\
        \end{bmatrix}
    \end{equation*}
    \item 同理,用公式(1.2)(1.3),计算Hidden Layer 到Input Layer的Error $E^2$和错误$\delta^{2}$:
    \begin{equation*}
        E^2 = \delta^3 \cdot (W^3)^T
        =\begin{bmatrix}
            -0.14758405\\
            0.08793035\\
            0.08805042\\
            -0.14749553\\
        \end{bmatrix} \cdot
        \begin{bmatrix}
            -0.09260631\\
            0.07315842\\
            0.79334259\\
            0.98067789\\
        \end{bmatrix} ^T
    \end{equation*}
    \begin{equation*}
        = \begin{bmatrix}
            0.01366721 & -0.01079702 & -0.11708471 & -0.14473241\\
            -0.00814291 & 0.00643285 & 0.06975889 & 0.08623135\\
            -0.00815402 & 0.00644163 & 0.06985415 & 0.0863491\\
            0.01365902 & -0.01079054 & -0.11701449 & -0.14464561\\
        \end{bmatrix}
    \end{equation*}
    \begin{equation*}
        \delta^2 = E^2 \circ \sigma^{'}(a^2)
    \end{equation*}
    \begin{equation*}
        =E^2 \circ
        \begin{bmatrix}
            \sigma^{'}(0.34506738) & \sigma^{'}(0.27875562) & \sigma^{'}(0.29619137) & \sigma^{'}(0.58874642)\\
            \sigma^{'}(0.31194502) & \sigma^{'}(0.13328488) & \sigma^{'}(0.2760579) & \sigma^{'}(0.31655429)\\
            \sigma^{'}(0.34076769) & \sigma^{'}(0.18305079) & \sigma^{'}(0.20476287) & \sigma^{'}(0.37164735)\\
            \sigma^{'}(0.3160554) & \sigma^{'}(0.20964762) & \sigma^{'}(0.38395048) & \sigma^{'}(0.52854125)\\
        \end{bmatrix}
    \end{equation*}
    \begin{equation*}
        =\begin{bmatrix}
            0.00308873 & -0.00217075 & -0.02440772 & -0.0350432\\
            -0.00174776 & 0.00074312 & 0.01394131 & 0.01865595\\
            -0.00183176 & 0.0009633 & 0.0113747 & 0.02016473\\
            0.00295259 & -0.00178794 & -0.02767773 & -0.03604357\\
        \end{bmatrix}
    \end{equation*}

    \item 有了$E^{l}$和$\delta^{l}$,就能用梯度下降更新权重矩阵$W^{l-1}$了。因为$l = 3$,所以我们要用梯度下降前的权重矩阵$W^2(old)$求更新后的权重矩阵$W^2(new)$:
    \begin{equation*}
        W^2(new) = W^2(old) + (z^{2}_{kj})^T \cdot \delta^3
    \end{equation*}
    \begin{equation*}
        = \begin{bmatrix}
            -0.09260631\\
            0.07315842\\
            0.79334259\\
            0.98067789\\
        \end{bmatrix} + 
        \begin{bmatrix}
            0.34506738 & 0.27875562 & 0.29619137 & 0.58874642\\
            0.31194502 & 0.13328488 & 0.2760579 & 0.31655429\\
            0.34076769 & 0.18305079 & 0.20476287 & 0.37164735\\
            0.3160554  & 0.20964762 & 0.38395048 & 0.52854125
        \end{bmatrix}^T \cdot
        \begin{bmatrix}
            -0.14758405\\
            0.08793035\\
            0.08805042\\
            -0.14749553\\
        \end{bmatrix}
    \end{equation*}
    \begin{equation*}
        = \begin{bmatrix}
            -0.13271534\\
            0.02893393\\
            0.73530181\\
            0.87638927\\
        \end{bmatrix}
    \end{equation*}
    \item 同样,计算从第二层(Hidden Layer)到第一层(Input Layer)之间的权重矩阵$W^1(new)$:
    \begin{equation}
        W^1(new) = W^1(old) + (a^1_{kj})^T \cdot \delta^2
    \end{equation}
    \begin{equation*}
        \text{, where  } \\
        W^1(old) = \begin{bmatrix}
            -0.01908238 & -0.54517074 & -0.49128704 & -0.88394168\\
            -0.13116675 & -0.37640824 & 0.39268698 & -0.24449632\\
            -0.64079264 & -0.95064254 & -0.86550074 & 0.35878555\\
        \end{bmatrix}
    \end{equation*}
    \begin{equation*}
        (a^1_{kj})^T = 
        \begin{bmatrix}
            0 & 0 & 1\\
            1 & 1 & 1\\
            1 & 0 & 1\\
            0 & 1 & 1\\
        \end{bmatrix}^T
    \end{equation*}
    \begin{equation*}
        \delta^2 = 
        \begin{bmatrix}
            0.00308873 & -0.00217075 & -0.02440772 & -0.0350432\\
            -0.00174776 & 0.00074312 & 0.01394131 & 0.01865595\\
            -0.00183176 & 0.0009633 & 0.0113747 & 0.02016473\\
            0.00295259 & -0.00178794 & -0.02767773 & -0.03604357\\
        \end{bmatrix}
    \end{equation*}
    所以(1.4)的结果为
    \begin{equation*}
        W^1(new)=\begin{bmatrix}
            -0.0226619 & -0.54346432 & -0.46597103 & -0.845121\\
            -0.12996192 & -0.37745306 & 0.37895056 & -0.26188394\\
            -0.63833084 & -0.95289481 & -0.89227018 & 0.32651946\\
        \end{bmatrix}
    \end{equation*}
\end{enumerate}
\end{document}
